\documentclass[12pt]{article}
\usepackage[landscape]{geometry}
\usepackage{url}
\usepackage{multicol}
\usepackage{amsmath}
\usepackage{esint}
\usepackage{bigints}
\usepackage{amsfonts}
\usepackage{xcolor}
\usepackage{tikz}
\usetikzlibrary{calc}
\usetikzlibrary{decorations.pathmorphing}
\usepackage{amsmath,amssymb}
\usepackage{setspace}

\usepackage{colortbl}
\usepackage{xcolor}
\usepackage{mathtools}
\usepackage{amsmath,amssymb}
\usepackage{enumitem}
\usepackage{xhfill}
\makeatletter

\newcommand*\bigcdot{\mathpalette\bigcdot@{.5}}
\newcommand*\bigcdot@[2]{\mathbin{\vcenter{\hbox{\scalebox{#2}{$\m@th#1\bullet$}}}}}
\makeatother

\title{física eléctrica}
\usepackage[brazilian]{babel}
\usepackage[utf8]{inputenc}
%cambios de unidades (pt, mm, cm,ex,em,bp,dd,pc,sp) https://tex.stackexchange.com/questions/8260/what-are-the-various-units-ex-em-in-pt-bp-dd-pc-expressed-in-mm
\advance\topmargin-.8in
\advance\textheight3in
\advance\textwidth3in
\advance\oddsidemargin-1.5in
\advance\evensidemargin-1.5in
\parindent0pt
\parskip2pt
\newcommand{\hr}{\centerline{\rule{3.5in}{1pt}}}
%\colorbox[HTML]{e4e4e4}{\makebox[\textwidth-2\fboxsep][l]{texto}
\newcommand{\nc}[2][]{%
\tikz \draw [draw=black, ultra thick, #1]
    ($(current page.center)-(0.5\linewidth,0)$) -- 
    ($(current page.center)+(0.5\linewidth,0)$)
    node [midway, fill=white] {#2};
}% tomado de https://tex.stackexchange.com/questions/179425/a-new-command-of-the-form-tex

\setstretch{1.5}
\begin{document}

\begin{center}{\huge{\textbf{FÍSICA II}}}\\
\end{center}
\begin{multicols*}{3}

\tikzstyle{mybox} = [draw=white, fill=white, very thick,
    rectangle, rounded corners, inner sep=10pt, inner ysep=13pt]
\tikzstyle{fancytitle} =[fill=white, text=white, font=\bfseries]

%---------------------------
\begin{tikzpicture}
\node [mybox] (box){%
    \begin{minipage}{0.3\textwidth}
    $F=kx$\\
    $x(t)=A\cos{(\omega t + \phi)}$ \\
    $T=\frac{2\pi}{\omega}$\\
    $A=\sqrt{x^2_1 + \frac{v^2_1}{\omega^2}}$\\
    $\omega=\sqrt{\frac{k}{m}}$\\
    $U=\frac{1}{2}kx^2$\\
    $T=2\pi\sqrt{\frac{L}{g}}$
    \end{minipage}
};
%---------------------------------
\node[fancytitle, right=10pt] at (box.north west) {MHS};
\end{tikzpicture}

%---------------------------
\begin{tikzpicture}
\node [mybox] (box){%
    \begin{minipage}{0.3\textwidth}
    $F_f=-bv$\\
    $x(t)=Ae^{-\beta t}\cos{(\omega_1 t + \phi)}$ \\
    $\beta = \frac{b}{2m}$\\
    $\omega_1 = \sqrt{\omega^2_0 - \beta^2}$\\
    $E(t)=E_0e^{-2\beta t}$\\
    $\tau = \frac{m}{b}$\\
    $Q=\frac{\omega_0}{2\beta}$
    \end{minipage}
};
%---------------------------------
\node[fancytitle, right=10pt] at (box.north west) {AMORTIDES};
\end{tikzpicture}

%---------------------------
\begin{tikzpicture}
\node [mybox] (box){%
    \begin{minipage}{0.3\textwidth}
    $F_0=amw^2$\\
    $A=\frac{F_0}{m\sqrt{(\omega_0^2-w^2)^2+4\beta^2\omega^2}}$ \\
    $\tau=\frac{1}{\beta} \text{   (de 3 a 7)}$\\
    $\tan{(\epsilon)}=\tan{\left(\delta-\frac{\pi}{2}\right)}=\frac{1}{-\tan{\delta}}=\frac{\omega^2-\omega_0^2}{2\beta\omega}$\\
    $Z=m\sqrt{\left(\frac{\omega_0^2-\omega^2}{\omega}\right)^2+4\beta^2}$\\
    $v(t)=\omega A\cos{(\omega t-\epsilon)}$\\
    $\omega = \sqrt{\omega_0^2-2\beta^2}$\\
    $\cos{\epsilon}=\frac{b}{Z}$\\
    $V_{RES}=\frac{F_0}{b}\cos{(\omega_0t)} \hspace{5mm} x_{RES}=\frac{F_0}{b\omega_0}\sin{(\omega_0t)}$\\
    $<P_S>=\frac{1}{2}\frac{F_0^2}{Z^2}b \hspace{10mm} <P_S>_{RES}=\frac{1}{2}\frac{F_0^2}{b}$\\
    $Q=\frac{\omega_0}{2\beta}=\frac{\omega_0}{\Delta\omega}$\\
    $\Delta\omega = 2\beta$
    \end{minipage}
};
%---------------------------------
\node[fancytitle, right=10pt] at (box.north west) {FORÇADES};
\end{tikzpicture}

%---------------------------
\begin{tikzpicture}
\node [mybox] (box){%
    \begin{minipage}{0.3\textwidth}
    $I=\frac{1}{3}ML^2$\\
    $I=\frac{1}{12}ML^2$ \\
    $I=\frac{1}{2}MR^2$\\
    $I=MR^2$
    \end{minipage}
};
%---------------------------------
\node[fancytitle, right=10pt] at (box.north west) {MOM. INÈRCIA};
\end{tikzpicture}

\end{multicols*}
\end{document}